\documentclass[a4,10pt,titlepage]{article}
\usepackage{taro}
\begin{document}
	\title{\textbf{Notes concerning article}}
	\author{Taro Valentin Brown}
	\maketitle
	\section{Role of $\gamma$}
	\subsection{Long version}
In the article the chiral mass $\chi$ transformed as $\chi\ra \lambda^y \chi$. The exponent $y$ is related to the mass anomalous dimension $\gamma_m$ which we will show here. 

We will work with the renormalized and bare mass $m_r$ and $m_0$. the renormalization scale is $\mu$. The renormalized mass is defined through
\be \label{eq:barem}
m_0\lp(\lambda \rp)\equiv m_r \lp(\lambda \rp) Z_m(\epsilon,g_r)
\ee
The mass anamolous dimension $\gamma_m$ is defined by
\be \label{eq:gammam}
- \gamma_m\equiv \pdv{\log m_r \lp(\lambda \rp)}{\log \mu}=-\pdv{\log Z_m}{\log \mu}=\pdv{\log Z_m}{\log \mu_0}
\ee
where $Z_m=Z_m(\frac{\mu}{\mu_0})$. The bare mass and renormalization scale are 
\[
m_0(\lambda)=\lambda m_0,\:\:\:\mu_0(\lambda)=\lambda \mu_0
\]
We then want to find how $m_r(\lambda)$ transforms. Differentiating equation \ref{eq:barem} wrt $\log \lambda$ gives
\[
\pdv{m_0\lp(\lambda \rp)}{\log \lambda}=\pdv{}{\log \lambda}\lp(m_r\lp(\lambda \rp) Z_m\rp)=Z_m\pdv{m_r \lp(\lambda \rp)}{\log \lambda}+m_r\lp(\lambda \rp)\pdv{Z_m}{\log \lambda}
\]
Dividing through by $Z_m$ we get
\be \label{eq:someeq}
\frac{1}{Z_m}\pdv{m_0\lp(\lambda \rp)}{\log \lambda}=\pdv{m_r\lp(\lambda \rp)}{\log \lambda}+\frac{m_r\lp(\lambda \rp)}{Z_m}\pdv{Z_m}{\mu_0}\pdv{\mu_0}{\log \lambda}
\ee
Both the $m$ and $\mu_0$ logarithmic derivatives go as:
\[
\pdv{m_0(\lambda)}{\log\lambda}=\pdv{m_0(\lambda)}{\lambda}\lp(\pdv{\log \lambda}{\lambda} \rp)^{-1}=m_0\lambda=m_0(\lambda)
\]
Using this on both sides of \ref{eq:someeq} we are left with
\be \label{eq:nextstep}
\frac{m_0(\lambda)}{Z_m}=\pdv{m_r\lp(\lambda \rp)}{\log \lambda}+\frac{m_r\lp(\lambda \rp)}{Z_m}\pdv{Z_m}{\mu_0}\mu_0(\lambda)
\ee
We want something of the form in \ref{eq:gammam}, so the $Z$ derivative is rewritten to include logarithms: 
\[
\pdv{Z_m}{\mu_0}=\pdv{\log Z_m}{\log \mu_0}\pdv{\log \mu_0}{\mu_0}\lp(\pdv{\log Z_m}{Z_m}\rp)^{-1}=-\gamma_m \frac{Z_m}{\mu_0}
\]
By plugging this into \ref {eq:nextstep} and using the relation between the renormalized and bare masses we get the differentialequation
\[
\frac{m_0(\lambda)}{Z_m}=m_r(\lambda)=\pdv{m_r\lp(\lambda \rp)}{\log \lambda}-\gamma_m m_r(\lambda)
\]
Which can be rewritten in the simpler form:
\be
\pdv{m_r(\lambda)}{\log \lambda}=\lp( 1+\gamma_m \rp)m_r(\lambda)
\ee
Solving this is done by rewriting the logarithmic derivative
\be
\pdv{m_r(\lambda)}{\lambda}=\frac{\lp( 1+\gamma_m \rp)}{\lambda}m_r(\lambda)\Rightarrow m_r( \lambda)=\lambda^{1+\gamma_m}m_r
\ee
which means that the the renormalized mass transforms as
\be
m_r\ra\lambda^{1+\gamma_m}m_r
\ee
\subsection{Short version}
By using a renormalization scale of $\mu$ and the usual way of defining a renormalized mass $m_r=\frac{1}{Z_m}m_0$ we define the mass anomalous dimension by
\be
\gamma_m\equiv \pdv{\log m_r}{\mu}
\ee
If the  bare mass transforms like its canonical dimension according to
\[
m\ra \lambda  m
\]
this leads to the renormalized mass transforming as
\[
m_r\ra \lambda^{1+\gamma_m} m_r
\]
since it also tranforms as 
\[
m_r\ra \lambda^{4-y}m_r
\]
$\gamma_m$ is related to $y$ by
\be
y=3-\gamma_m
\ee

\end{document}

